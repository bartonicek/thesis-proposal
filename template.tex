% Here is a suggested template for PhD research proposal for the
% first annual report.
% Written originally 2010-06-22 by T. W. Yee.
% Last modified      2018-06-15 by T. W. Yee.
% Last modified      2018-08-04 by T. W. Yee, incorporating
%                    sample-research-proposal-2.pdf from UoA.


\documentclass[12pt,a4paper]{article}


 \usepackage{natbib}    % For BibTeX
 \usepackage{graphicx}  % To import .pdf files
%\usepackage{times}
 \usepackage{amsfonts}
 \usepackage{amsmath}
 \usepackage{amssymb}

 \oddsidemargin  -10mm
 \evensidemargin -10mm
 \headheight 0mm
 \headsep -3mm
\textheight 250mm
\textwidth 180mm
\topmargin -4mm
\topskip -10mm

%\textwidth=450pt
%\hoffset=-2cm


\newcommand{\RR}{{\textsf{R}}}


\begin{document}

\begin{Large}
\begin{center}
\textbf{``Principled and Efficient Interactive Data Visualization''} \\
\textbf{by Bartonicek, Adam} \\
\textbf{for a PhD in Statistics}
\end{center}
\end{Large}


\hfill{Student ID: 828803059}

\hfill{Email: abar435@aucklanduni.ac.nz}

\hfill{Department of Statistics}

Supervisor: Dr.~Simon~Urbanek

Co-supervisor: Dr.~Paul~Murrell

Advisory Committee: Drs~B.~Efron, D.~R.~Cox, K.~Pearson.



\begin{center}
Date of enrolment in the programme and expected date of completion
\end{center}



\begin{center}
\today
\end{center}



This document represents the student's research proposal after
one year of provisional PhD registration.
Confirmed PhD registration is now sought.




% ----------------------------------------------------------------------
\section{Introduction}
\label{sec:intro}



Humans learn about the world around them by interacting with it. Further, as major part
of our cerebral cortex is devoted to visual processing, we learn best by interacting
with things we can see. The same applies to data. If we want to learn from our 
data effectively, we need practical and reliable tools for visualizing data and interacting with it.
As such, interactive data visualization is an important pursuit in statistics and data science, and this is evidenced by its rising popularity: currently, many interactive data visualization libraries exists across popular data-science-adjacent programming languages (such R, Python, Julia, and Javascript)
and their ecosystems, including D3 \citep{bostock2011}, plotly \citep{plotly2023}, Highcharts \citep{highcharts2023},
and Vega \citep{satyanarayan2015} and Vega-lite \citep{satyanarayan2016}.

Yet, many of these present-day interactive data visualization libraries libraries tend to suffer from a common set of drawbacks. Specifically, due to historical and other reasons, the libraries tend to fall into two camps: 

\begin{enumerate}

\item Specialized, high-level suites of "off-the-shelf" interactive visualizations
\item Generic, highly customizable low-level frameworks.

\end{enumerate}

 While the libraries from the first camp are usually approachable to users with less programming experience, their major drawback is that they are not extendable. Conversely, libraries from the second camp lend the user great deal of power and flexibility, however, to use them effectively, the users are required to have significant programming experience and deep knowledge of the specific API, especially when it comes to setting up complex interaction between multiple plots. Further, the users are responsible for making sure that the interaction is coherent - there is nothing stopping them from creating figures with interactive features that are meaningless, confusing, or unpredictable. Finally, the design philosophy behind many of these libraries seems to be oriented towards and used for data presentation rather than data exploration \citep{batch2017}. As such, within the interactive data visualization sphere, there is currently a lack of a mid-level, multi-plot interaction system geared towards data exploration.

Crucially, the abovementioned absence of a mid-level interactive data visualization system is not just an implementation gap. Rather, it seems to be a symptom of an underlying lack of a strong theoretical foundation. Specifically, there do not seem to be any guidelines for how such a mid-level system should operate - what criteria should interactive plots meet so that they can be composed together and behave in predictable and consistent ways. As a consequence, the two options left to the creators of interactive data visualizations systems are either: 1) give the user a limited number of ready-made solutions that are known to behave well, or 2) put the onus of making sure that the interaction is coherent onto the user - which precisely correspond to the two interactive data visualization camps outlined above. 

The goal of the proposed project is to lay down the theoretical foundations for such a mid-level interactive data exploration system, as well as implement it within a modern programming language. Specifically, at the theoretical level, I will map out the boundaries and constraints that parts of the system - underlying datastructures, plots, graphical primitives, etc... - need to meet in order to satisfy certain criteria of coherency and consistency, such that the user's interactions with the visualization are meaningful, understandable, and predictable. At the applied level, I will implement the system in \textbf{JavaScript} and provide a high-level interface in \textbf{R}. Finally, the broader teleological goal of the project is to create a tool that applied statisticians and data scientists can use to understand their data in a convenient, principled, and efficient way. 

% ----------------------------------------------------------------------
\section{Background}
\label{sec:background}

\subsection{What Event Counts as \textit{Interactive} Data Visualization?}
\label{sec:whatcounts}

It may seem surprising, but despite the widespread popularity of interactive data visualizations, there seems to be little consensus as to what actually makes a data visualization interactive. The term gets used by different researchers in vastly different contexts, sometimes in arguably conflicting ways. As such, for the purposes of the present text, it is important to disambiguate what is meant by "interactive data visualization".  

Firstly, there is the issue of whether "interactive data visualization" refers to an object or system, or to an action, undertaken by a human being. \cite{pike2009} note that 
"interaction" is an overloaded term that can refer to either the concrete tools which users use to manipulate visual information or to the more abstract "human interaction with information" - the back-and-forth between the user and the visual information presented to them \citep[see also][]{yi2007}. The more abstract action definition is more often emphasized in the field of Human Computer Interaction \citep[see e.g.][]{sinha2010}. For the purpose of this text, the object/system definition will be used - interactive data visualizations are concrete objects that are produced by (typically computer-based) systems/pipelines that take in raw data and produce images that can be interpreted and manipulated by humans \citep{brodbeck2009}.

\subsubsection{Simple Interactivity}

But even after narrowing down the focus to interactive data visualizations as objects, a lot of conceptual ambiguity remains. Some researchers use a \textbf{simple} definition and define interactive visualizations as any visualizations that can be actively manipulated by the user \citep{brodbeck2009}. Other researchers emphasize time or the \textbf{temporal} aspect, with visualizations being "interactive" when there is little lag between the user's input and changes to the visualization \citep{becker1987,buja1996}. Complicating the matters futher, some even make the distinction between "interactive" and "dynamic" manipulation, where interactive manipulation happens discretely, such as when pressing a button or selecting an item from a drop-down menu, whereas dynamic manipulation happens continuously, for example when smoothly moving a slider or by clicking-and-dragging \citep{rheingans2002,jankun2007model}. These tentative definitions present relatively little restriction on what counts as interactive visualization: for example, one could argue that the process of a user typing code into a command line to generate new plots could be considered interactive visualization, as long as it happens fast enough. 

\subsubsection{Complex Interactivity}

Other researchers do not seem to be satisfied with these broad definitions. For many, the defining feature of interactive data visualization is the ability to \textbf{query} different parts of the dataset (by e.g. zooming, panning, and filtering), and the reactive propagation of changes between connected or \textbf{"linked"} parts of interactive figures \citep{kehrer2012,buja1996,keim2002,unwin1999}. Similarly, in Visual Analytics (VA) research, a distinction is made between "surface-level" (or "low-level") interactions, which manipulate attributes of the visual domain only (e.g. zooming and panning), and \textbf{"parametric"} (or "high-level") interactions, which manipulate attributes of mathematical models or algorithms underlying the visualization \citep{leman2013,pike2009}. 

There are other ways that the term "interactive data visualization" has been used \citep[for more detailed taxonomies, see][]{yi2007}, however, short summaries of the key definitions mentioned above are presented in Table \ref{tab:definitions} and will be referred to later on in the text. What is important is that the different types of interactivity imply very different levels of programming complexity. For example, simply changing the color of a point or a bar in a plot, irrespective of anything else, might be implemented by changing an attribute of the underlying graphical primitive only - the point/bar does not need to know about what data it represents. However, if the change happens in response to (linked) brushing within a different plot, then there does need to be some way of tracking which cases of the data belong to the primitive. Likewise, changing the width of a histogram bar does not affect the graphical attributes of the bar only, but the underlying operation (binning) needs to be recomputed with respect to the new value of the parameter.  

\begin{table}[ht]
\caption{
Definitions of Interactive Data Visualization
}
\centering
\ ~~~~ \\
\label{tab:definitions}
\begin{tabular}{|p{2cm}|p{6cm}|p{8cm}|}
\hline
Name & Short Definition & Details \\
\hline

Simple & Change happens & User can manipulate the visualization in some way \\

Temporal & Change happens in real time & There is little lag between the user's input and changes to the visualization  \\

Querying & Change results from subsetting & The user can query different parts of the dataset, interaction is analogous to subsetting rows of the data (e.g. zooming, panning, and filtering) \\ 

Linked & Change propagates & Parts of the visualization are connected or “linked”, such that interaction with one part produces a change in another (e.g. linked brushing) \\

Parametric & Change reflects an underlying model & The user can manipulate the parameters of some underlying model (e.g. rotating principal axes in a PCA scatterplot, changing the width of a histogram bar) \\

(Cognitive) & Change is caused and perceived by a human & The user engages in a back-and-forth with the visual information presented to them \\

\hline
\end{tabular}
\end{table}

The conceptual ambiguity about what gets called "interactive" visualization matters because it leads to radically different implementations in software packages (see Sections \ref{sec:briefhistory} and \ref{sec:currentage}). For example, the \textbf{R Graph Gallery} page on \textbf{Interactive Charts} \citep{holtz2022} features several examples of interactive visualizations, however, none of them meet the linked and parametric definitions of interactivity outlined in Table \ref{tab:definitions}. Interactive data visualization systems that exist within the open source data visualization ecosystem differ significantly in the amount of features and flexibility they offer, as well as in how much responsibility or "house-keeping" for maintaining the interactive state they offload onto the user.     


\subsection{Brief History of Interactive Data Visualization in Statistics}
\label{sec:briefhistory}

\subsubsection{Static Visualization Goes Digital}

Static data visualization has a rich and intricate history \citep[see e.g.][]{dix1998,chen2008, friendly2021,young2011}. Briefly, for a long time, it was considered at best an auxiliary field, however, at the end of 1950's, a series of developments lead to a great increase in its prominence. Firstly, at the theoretical level, the work of Tukey (\citeyear{tukey1962,tukey1977}) and \cite{bertin1967} established data visualization as valuable discipline in its own right. Secondly, at the applied level, the development of personal computers \citep[see e.g.][]{abbate1999} and high-level programming languages, most notably FORTRAN in 1954 \citep{backus1978}, made production of figures easy and accessible to the wider public. Combined, these developments lead to a surge in the use and dissemination of data visualizations.

Final development in the field of static data visualization that is important to mention was the Grammar of Graphics introduced by Leland \cite{wilkinson2012}. Prior to Wilkinson's work, data visualization systems tended to come in two flavors: low-level ones, in which the users had to create visualizations from scratch using graphical primitives, and high-level ones, in which the users could select from a limited range of ready-made visualization types. Wilkinson, building upon the work of Bertin and Tukey, developed theory for a mid-level visualization system - Grammar of Graphics - which allows the users to specify a broad range of statistical graphics by declaratively combining abstract plot attributes such as aesthetics, scales, coordinates, and geometric objects \citep{wilkinson2012}. Grammar of Graphics has been successfully implemented in several software packages, most notably the popular \textbf{ggplot2} R package \citep{wickham2010} and the proprietary software \textbf{Tableau} \citep{tableau2023}. 

\subsubsection{Birth of Interactive Visualization}

As static visualization entered the computer age, interactive data visualization would not be left far behind. Early systems appeared in the 1960's and 1970's, and tended to be specialized for one specific task. For example, \cite{fowlkes1969} used interactive visualization to show how probability densities reacted to change of parameters and transformations, and \cite{kruskal1964} used interactive visualization to showcase his multi-dimensional scaling algorithm (a way of embedding objects within a common space based on pairwise distance measurements). The first "general-purpose" system was \textbf{PRIM-9} \citep{fisherkeller1974}, which allowed for exploration of high-dimensional data in scatterplots using projection, rotation, subsetting and masking. Later systems grew on to become even more general and ambitious. For example, \textbf{MacSpin} \citep{donoho1988} and \textbf{XGobi} \citep{swayne1998} provided features such as interactive scaling, rotation, linked selection (or "brushing"), and interactive plotting of smooth fits in scatterplots, as well as interactive parallel coordinate plots and grand tours.

Following the turn of the 21st century, interactive data visualization systems saw even scope and flexibility and also began to be integrated into general-purpose statistical computing software. The successor system to XGobi, \textbf{GGobi} \citep{swayne2003} was made to be directly embeddable in R. Java-based \textbf{Mondrian} \citep{theus2002} allowed for sophisticated linked interaction between many different types of plots including scatteplots, histograms, barplots, scatterplot, mosaic plots, parallel coordinates plots, and maps. Finally, \textbf{iPlots} \citep{urbanek2003} implemented a general framework for interactive plotting that was not only embedded in R but could be directly programmatically manipulated, and was later further expanded and made performant for big data in \textbf{iPlots eXtreme} \citep{urbanek2011}.

\subsubsection{Common Features of Statistical Systems}

The statistics-based interactive data visualization systems came in various forms, however, they generally tended to support features such as multiple ready-made plot-types, interaction between multiple plots with shared underlying data and state, and interactive manipulation of model parameters. They tended to be oriented towards scientific audience, with data exploration as the primary goal. Finally, they tended to be made to be directly embedabble and interoperable with general-purpose statistical computing software.      

\subsection{The Web and Current Age Interactive Data Visualization}
\label{sec:currentage}

\subsubsection{Web-native Interactivity}

The developments in interactive data visualization within the field of Statistics were paralleled by those within Computer Science. Most notably, the rise of Web technologies in the mid 1990's and the appearance of JavaScript in 1995 as a high-level general-purpose programming language for the Web \citep[for a description of the history, see e.g.][]{wirfs-brock2020}, created an extremely versatile platform for highly-reactive and portable applications. JavaScript was created with the explicit purpose of making the Web interactive, and the fast dissemination of standardized web-browsers meant that online interactive applications could be accessed by anyone, from anywhere. Interactive data visualization became just one of many technologies highly sought after within the fledgling Web ecosystem. 

The very early systems such as \textbf{Prefuse} in 2005 \citep{heer2005} and \textbf{Flare} in 2008 \citep{flare2020} relied on plugins (Java and Adobe Flash Player, respectively). However, in the early 2010's, several true Web-native JavaScript-based interactive data visualization systems emerged. The most prominent among these is \textbf{D3}.js \citep{bostock2011}. D3 is a broad and general framework for manipulating HTML documents using data and displaying it with visualizations. It is in popular use still to this day and has spawned a number of specialized, higher-level data visualization libraries that abstract away the details and provide an interface to commonly used statistical plots, such as the also very prominent \textbf{plotly}.js \citep{plotly2022}. Importantly, in the D3 model, D3's engine only renders graphics: interactivity is left the user. If a user wants to create a visualization that's interactive, she has to write JavaScript functions that take care of updating the underlying parameters. A radically different approach altogether was taken by \textbf{Vega} \citep{satyanarayan2015}. In Vega, all aspects of a visualization including interactivity are specified declaratively, as a Plain Old JavaScript Object (POJO). As such, Vega carries a large number of reactive primitives. Similar to D3 and plotly, Vega also spawned its own higher-level libraries in \textbf{Vega-lite} \citep{satyanarayan2016}, and \textbf{Altair} \citep{vanderplas2018}. A final popular interactive data visualization library in JavaScript is \textbf{Highcharts} \citep{highcharts2023}. Similar to Vega-lite, Highcharts is a high-level declarative framework in which plots are created based on a POJO specification object. 

\subsubsection{Common Features of Web-based Systems}

These contemporary web-based interactive data visualization systems offer a great deal of expressiveness, however, it does seem to come at a cost. Specifically, the low-level frameworks like D3 and Vega allow the users to create almost arbitrarily complex interactive figures, and even the higher-level frameworks like plotly, Vega-lite, Altair, and Highcharts still offer a great deal of customizability. Yet, a lot of code and programmer time is required to create complex interactive figures, even in these higher-level frameworks. As a result, most examples that show up on the web or on the libraries' own showcase pages typically feature only shallow interactivity within a single plot - such as zooming, panning, and pop-up labels - and examples of more complex multi-plot interaction such as linked brushing or cross-filtering are far less common. Further, the coherence of interactivity is left to the user: there is nothing stopping them from creating figures with meaningless interactive features. As such, it is hard to call the frameworks like Vega-lite, Altair, and plotly true mid-level systems, the way that ggplot2 can be called a mid-level system for static graphics.  

The target audience of the web-based systems is also very different from that of the statistical systems (discussed in Section \ref{sec:briefhistory}). Most real-world use examples come from online news articles and business/government dashboards, with considerably fewer appearing in scientific outlets. Likewise, the focus seems to be more on data presentation rather than data exploration - communicating findings once they have been found rather than discovering them in the first place. This seems to make sense given the design of these systems - it is worth it to create complex interactive visualizations when there is the guarantee that many people will see the result; however, conversely, it may not be economical to do so with $n = 1$ (i.e. when the user is the only person who will see the visualization). Data presentation and data exploration are both very important pursuits, however, it does seem that the market for interactive data visualizations for EDA is currently underserved, and this may explain why seemingly few data scientists use interactive visualizations to explore their data \citep{batch2017}.  

\subsection{Specialization vs. Generality}

To re-state the common thread from the previous sections, over the last 30 years or so, interactive data visualization has undergone a divergent evolution within two largely independent branches: statistical and web-based/computer science. Different goals and features were prioritized within each branch. The statistical branch focused on creating specialized systems for scientific data exploration. Users could easily create complex interactive figures by picking from a limited range of pre-made, "out-of-the-box" plots that were designed to behave coherently and consistently when composed together. Conversely, in the web-based branch, greater focus was put on generality and data presentation. Users were given a great deal of power and flexibility to create arbitrarily complex interactive figures from scratch, however, the onus for ensuring the interactivity was coherent and consistent was put on them.

The difference between the two branches represents a fundamental tension between specialization and generality. Specialized systems are easy to use but hard to extend; generic systems are extensible by definition but require time and effort to use effectively.  

\subsection{The Problem of Statistical Summaries}

Every data visualization has at least one graphical primitive or geometric object that is used to represent some statistical summary of the data. This is true for both static and interactive visualizations. For example, the position of a point on a scatterplot represents the values of the variables on the x- and y-axes. Typically, the statistical summary used in a scatterplot is identity, meaning that the x- and y-position of the point represents the raw values of those variables. As another example, the height of a bar in a histogram conventionally represents the number of cases within a binned range of the x-axis variable. 

In static data visualizations, we are free to compute any kind of statistical summary we like. However, this is not the case for interactive visualizations, which are subject to unique constraints and challenges.

\subsubsection{Computational Cost}

First of, there is the problem of computing resources. In static visualizations, we compute the summary only once, before we render the plot. In interactive visualizations, we may need to recompute the statistical summaries in response to the user's input. For example, if the width of the histogram bins changes, we need to recompute the number of cases within each bin. This incurs an additional computational cost. To clarify, the problem only arises when the interaction needs to refer back to the original data (i.e. when it is linked, querying, or parametric type of interaction, as described in Section \ref{sec:whatcounts}); no extra cost is incurred by e.g. interactively changing the opacity of graphical primitives, irrespective of the data. However, for more complex types of interaction, the necessity to recompute summaries can create a computational bottleneck. If the statistical summary is too computationally expensive, the volume of data is too high, or if the user's input can change too rapidly, it may not be possible to render the interaction smoothly enough.  

\subsubsection{Visual and Computational Coherency}

Perhaps more importantly, different types of interaction also place a limit on what summaries will be computationally and visually coherent. For example, suppose we have an interactive visualization that consists of a linked barplot and a scatterplot. Further, let's suppose that the barplot displays the means of some continuous variable, within the levels of the variable on the x-axis (this type of plot is also sometimes called "dynamite plot", when error bars are shown). We want the user to be able to perform linked brushing on the two plots in a reciprocal way - clicking-and-dragging to select points in the scatterplot should highlight parts of the corresponding bar or bars, and vice versa, selecting a bar should highglight the corresponding points.

We immediately run into several problems. First of all, how do we draw an empty selection? In a barplot of sums or counts, 0 is a meaningful default value, since the sum or count of a set with no elements is zero. However, the mean of an empty set is not defined. We could simply not draw the bar, but this will decohere the statistical summary from the visual representation: the absence of a bar may signal that either no cases are selected and the mean is undefined, or that there are selected cases and their mean is equal to the lower y-axis limit. 

Second, if there are multiple selections present within a single x-axis variable level, how do we draw them? In a sum or count bar, we can stack selected groups on top of each other, and the total height of the stacked bar is equal to the sum of the heights of the sub-bars. This is not the case for mean. The mean of the group means does not have to equal the grand mean, and there is no idiosyncratic way of visually combining multiple means together. We could draw the means side-by-side as separate bars (a technique called "dodging" in \texttt{ggplot2}), however, this complicates the two-way reciprocal nature of the linked brushing - the user can no longer simply select the single unambiguous stacked bar, but instead has to learn by experience whether they can select one of the dodged group bars individually or whether they have to select all bars jointly, or some combination of both. 

Finally, and thirdly, when computing the mean, we have to keep track of two quantities: running sum and running count. If we wanted to combine two means, we would need to have access to both, not just the computed values. In contrast, for sum and count we need to keep track of one value only and the values can be combined without reference to anything else. This is not a big problem with mean and the computation can still be easily parallelized, but other types of summaries may be more tricky.

\subsubsection{Need for Structure}

Overall, the picture that emerges is that, when it comes to interactive data visualization, some types of statistical summaries are better than others. Specifically, when reciprocal, two-way linked brushing is desired, not every kind of statistical summary will do. It is important to map out what types of constraints should statistical summaries meet in order to "work" and lead to visually and computationally coherent visualizations. Only by imposing some kind of structure can we hope to lay the foundations for a true mid-level interactive data visualization system. 


\subsection{Few Relevant Bits of (Applied) Category Theory}

Category theory is a branch of mathematics concerned with the study of universal structures and relations. It was developed as a way of unifying historically distinct areas of math such as abstract algebra, topology, and linear algebra. More importantly, at the present time, it is also becoming increasingly used in applied fields, especially computer science and functional programming. Even a very basic treatment would be outside the scope of the present proposal, however, few relevant pieces will be presented in a greatly simplified way, introducing examples from programming when relevant. The treatment follows mainly from \cite{fong2019} and \cite{Milewski2018}, as well as \cite{leinster2014}.

\subsubsection{Functions}

Functions are the most fundamental building block of category theory (as well as many other areas of mathematics). They are also known, in their more general form, as "morphisms" or "mappings". A function can be described in the following way: given two sets, the set of sources $S$ and the set of targets $T$, a function is a subset $F \subseteq S \times T$ containing all source-target pairs $(s, t)$, such that for all $s \in S$ there exists a unique a $t \in T$. The sets $S$ and $T$ are also known as the domain and codomain, respectively. In programming terms, we should be hypothetically able to implement any function as a lookup table (in practice, this is only feasible when the number of possible arguments is finite and small - but then it can be handy technique, called \textit{memoization}).  

A function which covers all of its codomain, i.e. one for which, for all $t \in T$, there exists an $s \in S$ such that $f(s) = t$ is called \textit{surjective} or \textit{onto}. A function which has a unique element in the codomain for every element in the domain, i.e. one for which for all $s_1, s_2 \in S$ and $t \in T$, if $f(s_1) = t$ and $f(s_2) = t$, $s_1 = s_2$, is called \textit{injective} or \textit{one-to-one}. Also, for any given subset $T_i \subseteq T$, we can define \textit{pre-image} as the subset of $S$ that maps into $T_i$: $f^{-1}(T_i) = \{ s \in S : f(s) \in T_i \}$. Finally, functions can be composed: if we have two functions $f: X \to Y$ and $g: Y \to Z$, we can combine them into a new function $h = g \circ f$ such that $h: X \to Z$, i.e. $h(x) = g(f(x))$. 

\subsubsection{Partitions and preorders}

Two very simple structures used in category theory (and elsewhere) are partitions and preorders. Firstly, for a set $A$, a \textit{partition} of $A$ consists of a set of part labels $P$, such that, for all $p \in P$, there is a non-empty subset $A_p \subseteq A$ and:

$$A = \bigcup_{p \in P} A_p \qquad \text{and} \qquad \text{if } p \neq q, A_p \cap A_q = \varnothing \qquad (\forall p, q \in P)$$

I.e. the parts jointly cover the entirety of $A$ and their elements do not overlap. Partitions can also be conceptualized as surjective functions: given the two sets $A$ and $P$ described above, a surjective function $f : A \to P$ allocates every element in $A$ a part label in $P$, with the pre-images $f^{-1}(p) \subseteq A$ playing the role of the parts.  

Secondly, a \textit{preorder} consists of a set $X$ and a binary relation on $X$, often denoted $\leq$, such that:

\begin{enumerate}
\item $x \leq x$ (reflexivity)
\item if $x \leq y$ and $y \leq z$ then $x \leq z$ (transitivity)
\end{enumerate}

Further, if we have one additional property:

\begin{enumerate}
\setcounter{enumi}{2}
\item If $x \leq y$ and $y \leq x$, then $x = y$ (anti-symmetry)
\end{enumerate}

Then we can speak of a partially ordered set, or \textit{poset}. Examples of posets include the set of booleans $\mathbb{B} = \{ \text{True, False} \}$, natural numbers $\mathbb{N}$, reals $\mathbb{R}$, etc... 

% ----------------------------------------------------------------------
\section{So What's New?}
\label{sec:whatsnew}

This project seeks to develop a theory for a mid-level interactive data exploration system and fill its unoccupied niche within the interactive data visualization sphere. To accomplish this, the following goals will be actioned:  



\begin{enumerate}

\item
Map out boundaries and constraints that a mid-level interactive data visualization system should meet
  \begin{enumerate}
  \item Describe what contracts do graphical primitives and their underlying statistical summaries need to uphold in order to be interactable with in a coherent and predictable way 
  \end{enumerate}

\item
Develop a theoretical foundation for the system
  \begin{enumerate}
  \item Use concepts and syntax of category theory to describe the necessary structure
  \item Consider links back to implementation via functional programming
  
  \end{enumerate}

\item Implement the system in JavaScript and publish as an \texttt{npm} package
  \begin{enumerate}
  \item Use plain ES2020 as much as possible, possibly incorporate a reactive programming library and functional programming utility library
  \item Use the HTML \texttt{canvas} element as a drawing utility, explore using WebGL and other frameworks for performance
  \item Provide a full documentation for any functions, classes, etc...
  
  \end{enumerate}

\item Implement a high-level interface to the system in R and publish on \texttt{CRAN}
  \begin{enumerate}
  \item Provide a full documentation for all functions
  \item Provide a package vignette
  \item Pass all \texttt{CRAN} checks
  \end{enumerate}
  
\item Publish an article in a international peer-reviewed journal such as the \textit{Journal of Statistical Software} or the \textit{R journal}

\item Present the system at an R developer or data visualization conference such as useR! or IEEE VIS

\end{enumerate}


% ----------------------------------------------------------------------
\section{Data and Special Needs}
\label{sec:data}

The use of the system will be tested out on datasets available in the public domain such as the \texttt{diamonds} dataset in R. For these data sets, no ethics or disclosure statements are necessary. Possible collaboration with working scientists and corresponding data access may be negotiated at a later date.  

The student's own personal computer should be sufficient to produce the developing and testing the system. Should extra computing resources become necessary, the student should be able to inquire and be granted access to the \textbf{Ihaka server} (\texttt{ihaka.stat.auckland.ac.nz.}) using SSH connection. Other options such as the New Zealand eScience Infrastructure (\textbf{NeSI}) should be available too. 

The packages produced as part of the project will be published for free as open-source software, under the MIT license \citep{mit2023}. No additional ethics approvals are necessary. 

% ----------------------------------------------------------------------
\section{Budget}

Development of open-source software is not very budget-intensive. The work related to the project will be funded by the University of Auckland Doctoral Scholarship. Expenses related to conferences (registration fees, travel) may be funded via the Postgraduate Research Student Support (PReSS) account, with an annual allocation of \$1200 NZD. 

% ----------------------------------------------------------------------
\section{Objectives and Goals}



The objective(s) of this research project are to\ldots



List \textit{specific} goals/tasks that
will be undertaken as part of the proposed research.



\begin{enumerate}

\item
I wish to derive the first three moments of the Slash distribution.
This will involve working out the characteristic function first.


\begin{itemize}

\item[a.]
Then work out the observed information matrix
based on~\cite{bick:etal:2009}.

\item[b.]
Then work out the expected information matrix
\citep{scot:lee:wild:2007}.

\item[c.]
Apply the EM algorithm \citep{demp:lair:rubi:1977} to estimate~$\lambda$.

\item[d.]
Write a \RR{} package to implement my method.
It will be written in S4 and use object oriented methods
\citep{cham:1998}.

\item[e.]
Apply the method to the radiation data set
\citep{MR2526777}.

\item[f.]
Extend biplots \citep{gowe:1966} for my data.

\item[g.]
Release the package on CRAN
(cf.~\cite{murr:ihak:2000}).

\item[h.]
Publish my results in at least two papers,
earmarked for JASA and JRSS-B.
Also an applications paper in Biometrics.
See Section~\ref{sec:timeline}.

\end{itemize}



\item
Solve the Fisher-Behrens problem
\citep{efro:2009,MR2415600,MR2508377}.


\end{enumerate}







% ----------------------------------------------------------------------
\section{Deliverables and Program Schedule}
\label{sec:timeline}




\begin{table}[hh]
\caption{
Timeline for my thesis.
Itemize the list of deliverables with specific dates so that
you can make concerted effort to achieve them.
Here are \textit{some} activities---fill in more.
}
\centering
\ ~~~~ \\
\label{tab:timeline}
\begin{tabular}{|c|l|}
\hline
Date & Activity \\
\hline
2018-04-01 & Provisional PhD registration (PhD in Statistics). \\
2018-05-01 & (Optional)
             Updated my personal webpage at
             \textsf{www.stat.auckland.ac.nz/$\sim$myStudentName}. \\
2018-05-20 & Attended one of the
             Doctoral Skills Programme's Induction Days. \\
2018-09-01 & Gave my first talk at NZSA conference. Won first prize. \\
2018-11-01 & Gave a talk to PhD Talks Day. \\
2019-01-15 & Submit my first paper to \textit{Annals of Statistics}
             (co-authored with supervisor). \\
2019-05-01 & Presented my research progress to a departmental seminar. \\
2020-01-20 & Achieve Goal~1. \\
2020-08-20 & Achieve Goal~2. \\
2020-10-23 & Submit my second paper to \textit{Annals of Applied Statistics}
             (co-authored with supervisor). \\
2020-01-15 & Submit my third paper to \textit{Biometrics}
             (sole authorship). \\
2020-02-01 & Submit my PhD thesis. \\
\hline
\end{tabular}
\end{table}



I have fulfulled all my first year requirements.
These are:
\begin{enumerate}

\item
All AFA provisional goals:
\begin{enumerate}

\item
\textit{Full thesis proposal normally completed within 6 months}:
Here it is!

\item
\textit{Completion of one substantial piece of written work within
12 months}: Wrote 90~percent of Chapter~1 of my thesis, submitted
one journal article, wrote 50~percent of Chapter~2 of my thesis.


\item
\textit{Presentation of research progress to a departmental seminar}:
2019-05-01.

\item
\textit{Approval of the full thesis proposal by the appropriate
departmental/faculty postgraduate committee}:
Being done.

\item
\textit{Ethics approval(s)/permissions obtained for the research
(if required)}:
Not necessary.

\item
\textit{Attendance at one of the Doctoral Skills Programme's
Induction Days}:
2018-05-20.

\end{enumerate}



\item
Although optional,
it is a good idea to try update a personal webpage
(ideally \textsf{www.stat.auckland.ac.nz/$\sim$myStudentName})
giving details of my thesis, links to other research resources
in my topic, and some personal stuff to make it interesting.



\item
I have diligently attend as many Statistics Department seminars
as I could. They are given in Table~\ref{tab:seminars}.
This is much more than the minimum quota set by the department.
Consequently there should be no problem due to this when getting
my annual report signed off\footnote{If insufficient seminars have
been attended then sign off will occur \textit{after} the minimum number
is reached.}.
The 2018-10-08 seminar was particularly useful because it gave
me an idea on how to solve one of my problems.



\item
I did STATS~730 in Semester~1 of 2018 and obtained an A+.



\item
I did STATS~782 in Semester~2 of 2018 and obtained an A.



\end{enumerate}











\begin{table}[hh]
\caption{
Departmental seminars I have attended (top part of the table).
Talks from another UoA department are in the middle tier.
Conferences and workshops are in the bottom tier,
e.g., 2019-05-29 event was an all-day workshop.
\textbf{Note}:
I filled in the required document and submitted it
within the required time period after
each seminar I attended.
}
\centering
\ ~~~~ \\
\label{tab:seminars}
\begin{tabular}{|c|l|l|}
\hline
Date & Speaker & Title \\
\hline
2018-04-08 & David Brillinger &
Random trajectories, some theory and applications \\
%
2018-02-30 & David Cox &
Frequentist statistics as a theory of inductive inference \\
%
2018-10-08 & David Matthews &
Estimating diagnostic test likelihood ratios \\
%
yyyy-mm-dd & Speaker Name &
Title of Talk \\
%
yyyy-mm-dd & Speaker Name &
Title of Talk \\
%
yyyy-mm-dd & Speaker Name &
Title of Talk \\
%
yyyy-mm-dd & Speaker Name &
Title of Talk \\
%
yyyy-mm-dd & Speaker Name &
Title of Talk \\
%
yyyy-mm-dd & Speaker Name &
Title of Talk \\
%
yyyy-mm-dd & Speaker Name &
Title of Talk \\
%
yyyy-mm-dd & Speaker Name &
Title of Talk \\
%
yyyy-mm-dd & Speaker Name &
Title of Talk \\
%
yyyy-mm-dd & Speaker Name &
Title of Talk \\
%
\hline
2018-06-16 & James B.~Conant &
``Geography is not a university subject'' (geo Department) \\
%
yyyy-mm-dd & Speaker Name &
Title of Talk \\
\hline
2019-05-29 & David Siegmund &
Workshop on `Genetic Mapping' at UoA. \\
%
\hline
\end{tabular}
\end{table}








% ----------------------------------------------------------------------
\section*{Appendix~A}

Delete or replace the contents of this section with any
appendices you may have.

\bigskip

\bigskip

The \textit{University of Auckland Statute and Guidelines for the
Degree of Doctor of Philosophy (PhD)} (2016)
reads\footnote{Regulation~1, Preamble, item~(e).}:

\bigskip


\noindent
``The PhD degree is awarded for a formal and systematic
exposition of a coherent programme of advanced research
work. The work is  carried out over the period of enrolment
for the degree, and in the opinion of the examiners and the Board
of Graduate Studies, satisfies all of the following criteria:

\begin{itemize}

\item[(i)] is an original contribution to knowledge or
understanding in its field, and

\item[(ii)] meets internationally recognised standards
for such work, and

\item[(iii)] demonstrates knowledge of the literature
relevant to the subject and the field or fields to which
the subject belongs, and the ability to exercise critical
and analytical judgement of it, and

\item[(iv)] is satisfactory in its methodology, in the
quality and coherence of its expression, and in
its scholarly presentation and format.''

\end{itemize}





\addcontentsline{toc}{section}{References}
\bibliographystyle{./elsart-harv} % elsart-harv,plain,unsrt,alpha
\bibliography{./references}



\end{document}


